%% start of file `template.tex'.
%% Copyright 2006-2013 Xavier Danaux (xdanaux@gmail.com).
%
% This work may be distributed and/or modified under the
% conditions of the LaTeX Project Public License version 1.3c,
% available at http://www.latex-project.org/lppl/.


\documentclass[11pt,a4paper,sans]{moderncv}        % possible options include font size ('10pt', '11pt' and '12pt'), paper size ('a4paper', 'letterpaper', 'a5paper', 'legalpaper', 'executivepaper' and 'landscape') and font family ('sans' and 'roman')

% moderncv themes
\moderncvstyle{casual}                             % style options are 'casual' (default), 'classic', 'oldstyle' and 'banking'
\moderncvcolor{blue}                               % color options 'blue' (default), 'orange', 'green', 'red', 'purple', 'grey' and 'black'
%\renewcommand{\familydefault}{\sfdefault}         % to set the default font; use '\sfdefault' for the default sans serif font, '\rmdefault' for the default roman one, or any tex font name
%\nopagenumbers{}                                  % uncomment to suppress automatic page numbering for CVs longer than one page

% character encoding
\usepackage[utf8]{inputenc}                       % if you are not using xelatex ou lualatex, replace by the encoding you are using
%\usepackage{CJKutf8}                              % if you need to use CJK to typeset your resume in Chinese, Japanese or Korean

\usepackage{ragged2e}

% adjust the page margins
\usepackage[scale=0.75]{geometry}
%\setlength{\hintscolumnwidth}{3cm}                % if you want to change the width of the column with the dates
%\setlength{\makecvtitlenamewidth}{10cm}           % for the 'classic' style, if you want to force the width allocated to your name and avoid line breaks. be careful though, the length is normally calculated to avoid any overlap with your personal info; use this at your own typographical risks...

% personal data
\name{Alejandro}{Gómez-Londoño}
\title{Personal Letter}                               % optional, remove / comment the line if not wanted
\address{Oxhagsgatan 56}{Göteborg}{Sweden}
\phone[mobile]{+46-733-424-128}
\email{agomezl@refl.xyz}

% to show numerical labels in the bibliography (default is to show no labels); only useful if you make citations in your resume
%\makeatletter
%\renewcommand*{\bibliographyitemlabel}{\@biblabel{\arabic{enumiv}}}
%\makeatother
%\renewcommand*{\bibliographyitemlabel}{[\arabic{enumiv}]}% CONSIDER REPLACING THE ABOVE BY THIS

% bibliography with mutiple entries
%\usepackage{multibib}
%\newcites{book,misc}{{Books},{Others}}
%----------------------------------------------------------------------------------
%            content
%----------------------------------------------------------------------------------
\begin{document}
%-----       letter       ---------------------------------------------------------

\recipient{Klarnauts}
          {@Klarna's Göteborg office}
\date{\today}

\opening{}

\closing{Sincerely,}

\makelettertitle

\justify

My name is Alejandro Gomez Londoño. I have a BS in Computer Science
from EAFIT University (Medellin, Colombia). I am currently a PhD
student in formal methods at Chalmers University of Technology
(Gothenburg, Sweden). During the course of my bachelors, I was
involved in different projects with topics ranging from HPC
infrastructure to dependently typed functional programming. After
graduation, I worked as a proof engineer in developing and maintaining
large-scale computer-assisted proofs. This experience motivated me to
pursue a PhD in formal verification with an emphasis in compilers and
concurrent systems. Through this time, I have been exposed to many
practical and theoretical problems and have had lots of fun solving
them.

I have always had a passion for exploring new things, even if they
seemed challenging at the time.
%
Early in my bachelor's, I learned functional programming (Haskell) by
myself out of curiosity, using it for courses' projects as my go-to
language. This interest in functional programming led me to take
various advanced courses and began my path towards formal verification
research.
%
I also had the opportunity to work as part of the initial system
administration team for the HPC centre on campus. Our team deployed,
managed, and maintained most aspects of the infrastructure. We faced
several challenges, we were sometimes stuck, very often cold (from the
cluster's AC), but we persevered and were never bored. This experience
taught me many valuable lessons and helped me develop skills (build
systems, networking, GNU/Linux know-how, etc.) I still use in my work
today.

Shortly after graduation, I moved to Australia to work as a proof
engineer at Data61@CSIRO. I was part of a group tasked with
maintaining and developing proofs for the seL4 micro-kernel.  A high
performance OS-kernel for safety critical systems with formal
guarantees. Our team had to coordinate with others outside our area of
expertise, using code/proof reviews and issue tracking as ways to
manage progress between groups. Overall, my work at Data61 was
gratifying and motivated me to pursue a PhD in formal verification.
%
During my PhD, I worked on a language for concurrency with
deadlock-freedom guarantees. Additionally, I extended the verified
CakeML compiler (an SML-like language) to reason about a program's
memory consumption. As a PhD, I learned how to rigorously search for
answers to loosely defined questions, improve upon my failures, and,
more importantly, communicate the new bits of knowledge I was able to
produce with my peers.

Finally, I firmly believe that working at zenseact in Highway Pilot is
an excellent opportunity to apply my skills in interesting ways, face
new challenges, and meet new people. I still have many things to
learn, but I’m passionate and eager to do so.

Please have a look at my enclosed resume for more detailed information
regarding my work experience and education, and let me know if you
have any further questions. I look forward to discussing more in
person.

\makeletterclosing

\end{document}


%% end of file `template.tex'.
